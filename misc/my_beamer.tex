% My Beamer by Y.Chubachi (2016)

% * 色の設定
% - texdoc xcolorで色見本が確認できる(dvipsnames)
\colorlet{ThemeColor}{Green!70!white} % 基調となる色
\colorlet{AlertedText}{orange}        % 強調(alert)の色

% * Beamerのテーマ
% \usetheme{default}
% \usetheme{Szeged} % 横線
\usetheme{Madrid}
% \usetheme{Goettingen} % サイドバーの色が変わらない
% \usetheme[hideothersubsections]{Berkeley}

% * インナーテーマ
\useinnertheme{circles} % 箇条書きをシンプルに
\useinnertheme{rounded}
\setbeamertemplate{blocks}[rounded] % 影を消す(オプションで消せない?)

% * アウターテーマ
\useoutertheme[height=1.5cm, width=2cm, right]{sidebar}
\setbeamercolor{frametitle}{fg=white} % heightを設定すると黒になる
\setbeamertemplate{navigation symbols}{} % ナビゲーションシンボルを消す
% \setbeamertemplate{footline}[frame number] % フッターはスライド番号のみ

% * 色のテーマ
\usecolortheme[named=ThemeColor]{structure} % 基調となる色を設定
\setbeamercolor{alerted text}{fg=AlertedText} % 強調(alert)の色を設定

% * フォントテーマ
\usefonttheme{structurebold}
\renewcommand{\kanjifamilydefault}{\gtdefault}  % 本文の和字もゴシック体に

% 節ごとの目次フレーム
\AtBeginSection[]{
  \begin{frame}
    \frametitle{目次}
    \tableofcontents[currentsection,currentsubsection]
  \end{frame}
}

%% * フレームのカスタマイズ
% フレームタイトルのサイズを少々小さめに
\setbeamerfont{frametitle}{size=\large}

% 参考
% - Beamer — Tasuku Soma's webpage
% http://www.opt.mist.i.u-tokyo.ac.jp/~tasuku/beamer.html#tips
